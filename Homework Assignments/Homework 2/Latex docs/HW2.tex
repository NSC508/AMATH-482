\documentclass[11pt]{amsart}

% Standard letter size paper with 1inch margins
\usepackage[letterpaper, margin=1in]{geometry}
\usepackage{float}
% Useful packages 
\usepackage{amsmath, amssymb, amsthm, amsaddr}
\usepackage{enumerate, subcaption, graphicx, hyperref}


\title{AMATH 482/582: Homework 2}
\author{Sathvik Chinta} % first and last name

\date{\today} % you can also just type the date instead of "\today"

\begin{document}

\maketitle 

\begin{abstract}
    Using the Fast Fourier Transform algorithm alongside a Guassian Noise filter, we were able to succesfully
    de-noise acoustic data of a submarine located within Puget Sound and track its position over a 24 hour
    period of time. 
    % Your report should contain a brief, 100 word abstract describing what is contained in 
    % the document and what you did. {\bf Don't forget 6 pages max}.
\end{abstract}


\section{Introduction and Overview}\label{sec:Introduction}
This time, we have a very interesting (and fun) problem to solve. We want to create a model that can identify hand-written digits!
The data itself is already split into training and test data, so we won't have to worry about that. Taking a look at the data itself, 
we can see that there are 2000 images in the train data and 500 in the test, each with 256 modes. Knowing that we most likely won't need to use 
all the modes to represent our data, we can think about using PCA analysis in order to figure out how many modes we truly need.


% Here you will give a brief introduction to the problem you solved. Including 
% some discussion of relevant literature and background. 

% Make sure you use the correct citation commands (i.e., \texttt{$\backslash$cite}) to keys 
% from your bib file like this \cite{example-article-citation}. If you want 
% to cite more than one reference simply use \cite{example-article-citation, example-book-citation}. You can grab latex citations 
% from \href{https://scholar.google.com}{Google Scholar}. Just keep in mind that they often 
% need to be cleaned up.

\section{Theoretical Background}\label{sec:theory}
Imagine we are given some dataset in a very high dimensional space. We want to apply a model to identify classes on said dataset, 
but the complexity of multiple higher dimensions makes this task very difficult. This is where PCA analysis comes in. We check the covariance 
of each dimension to the other dimensions and then find the eigenvalues and eigenvectors of the covariance matrix. We then sort the eigenvalues in 
descending order and use the corresponding eigenvectors to project our dataset into a lower dimensional space. This is called PCA analysis. It allows us 
to also know the variance of each dimension and how much of the variance is explained by each dimension. So, if we know that the variance 
of some dimension is 20 percent of the total variance, we can say that 20 percent of the data is explained by that dimension. Knowing this, we can remove 
the dimensions that are not important to our model. This is called dimensionality reduction. Our current dataset has 
256 modes of data, so we should perform PCA analysis and see how many of these dimensions we can remove and still have a good model.

Once we have done so, we can use Ridge regression to find the best weights for our model. We can then use the weights to predict the class of a new
data point within our dataset. In Ridge regression, we estimate the weights and assume that the dimensions are highly correlated with each other. This is 
why it's so important to perform PCA analysis before hand. In ridge regression, we attempt to minimize the following equation:

\[minimize_\beta ||A\beta - Y||^2 + \lambda||\beta||^2\]

Where A is our prediction, y is the true value and $\lambda$ is our regularization parameter. 
% You dedicate this section to the theoretical background of the methods and frameworks 
% that you used in your homework. This is not meant to reproduce material from the lectures
%  or references you used but rather to demonstrate your understanding of the 
%  mathematical foundations of the methods and algorithms. You can create equations like this 
%  \begin{equation*}
%      f(x) = \int_A \sin( \pi x) dx.
%  \end{equation*}
%  You do not need to label your equations unless they are referenced in the text. In that 
%  case simply use 
%  \begin{equation}\label{eq:meaningful-label}
%       - \frac{\partial^2 u}{\partial x^2} = \sin ( \pi x).
%  \end{equation}
% Also look up the \texttt{align} or \texttt{aligned} environments if you want multi-line 
% equations. You can then reference your equations in text using the $\backslash$\texttt{eqref}
% command as such \eqref{eq:meaningful-label}. 

\section{Algorithm Implementation and Development}\label{sec:algorithms}
I used Python for this project. The major package that I used was sklearn, but I also used numPy and matplotlib. 

Within sklearn I used the following functions:
\subitem \texttt{sklearn.decomposition.PCA()} to perform PCA analysis.
\subitem \texttt{sklearn.singular values} to find the singular values of the data.
\subitem \texttt{sklearn.decomposition.PCA().fit()} to fit the PCA model to the training data.
\subitem \texttt{sklearn.linearModel.Ridge()} to initialize the ridge regression model.
\subitem \texttt{sklearn.metrics.meanSquaredError()} to find the mean squared error of the model.
\subitem \texttt{sklearn.linearModel.Ridge().fit()} to fit the ridge regression model to the training data.
\subitem \texttt{sklearn.linearModel.Ridge().predict()} to predict the class of a new data point.

Within numPy I used the following functions:
\subitem \texttt{np.load()} to load in both the train and test data.
\subitem \texttt{np.cumsum()} to get the cumulative sum while looking at the total variance explained by each dimension.

Within matplotlib I used the following functions:
\subitem \texttt{plt.plot()} to plot all graphs/data


% Here you discuss the algorithms and software packages that you used. Not much to it. 
% Just make sure you cite the packages properly and avoid including code. 
% You are welcome to use \LaTeX packages that are specifically designed to show 
% algorithms such \href{https://www.overleaf.com/learn/latex/Algorithms}{as this}, but it is 
% not always worth the effort and real estate. 


\section{Computational Results}\label{sec:results}



% This is perhaps the most important section of your report. You want to dedicate more space 
% here and present your numerical results in a clear, concise and meaningful way. Also 
% include a discussion of your numerics. Think hard about how you can use 
% your space most efficiently. For example, include subplots and multiple error curves on the 
% same plot etc. Ask us for advice when the time comes. 

% You will most definitely need tables and figures. So here is an example. 

% \begin{table}[htp]
%     \centering
%     \begin{tabular}{| l | c|c | r |}
%          \hline
%          row 1 & column 1  & column 2  \\ \hline
%          row 2 & column 1 & column 2 \\ 
%          row 3 & column 1 & column 2 \\ \hline
%     \end{tabular}
%     \caption{Don't forget to include a caption for your table. Say a few words about what is 
%     being shown.}
%     \label{tab:meaningful-label}
% \end{table}

% Make sure your table is labeled and referenced withing the text using $\backslash$\texttt{ref} as such Table~\ref{tab:meaningful-label}. In fact, you can 
% use $\backslash$\texttt{ref} to cite anything else in the document such as 
% sections (ex. Section~\ref{sec:Introduction}). This will create hyperlinks in your 
% pdf after compilation and automatically update the numbers and tags whenever you change 
% anything. 

% Figures are very similar to tables. Here's an example: 

% % \begin{figure}[htp]
% %     \centering
% %     \includegraphics[width=0.4\textwidth]{./Figs/fig1.pdf}
% %     \caption{Include a descriptive caption for your figure. Also make sure all 
% %     legends, axis labels, and titles are large enough to be readable. You might have 
% %     to reproduce the plots from Python or MATLAB with larger fonts for this purpose. It 
% %     can be annoying the first time you do it but it is crucial.}
% %     \label{fig:meaningful-label}
% % \end{figure}

% You may also need to include multiple figures: 

% % \begin{figure}
% %     \centering
% %     \begin{subfigure}[b]{.3\textwidth}
% %     \includegraphics[width=\textwidth]{./Figs/fig1.pdf}
% %     \caption{First subfigure}
% %     \label{subfig:first}
% %     \end{subfigure}
% %     \begin{subfigure}[b]{.3\textwidth}
% %     \includegraphics[width=\textwidth]{./Figs/fig2.pdf}
% %     \caption{First subfigure}
% %     \label{subfig:second}
% %     \end{subfigure}
% %     \begin{subfigure}[b]{.3\textwidth}
% %     \includegraphics[width=\textwidth]{./Figs/fig3.pdf}
% %     \caption{First subfigure}
% %     \label{subfig:third}
% %     \end{subfigure}
% %     \caption{Caption for entire figure. You don't need to use captions for subfigs so 
% %     feel free to eliminate the subcaption texts to just have the A, B, C labels.}
% %     \label{fig:meaningful-label-2}
% % \end{figure}

% Once again, make sure all your figures are referenced like Figure~\ref{fig:meaningful-label}
% or Figure~\ref{subfig:first} in the text body of the report and discussed 
% in detail. This is where you will make observations about your results and we will 
% look at these very closely. 

% Also note, I am using PDF figures. These give you the best looking graphs but PNG works 
% well too. I advise staying away from JPG as it always looks weird and low quality.]
% Both Python and MATLAB can output figures in PDF or PNG.

\section{Summary and Conclusions}\label{sec:conclusions}

% Wrap up your report with a brief summary of what you did and what you discovered. 
% Finish with some conclusions and possibly future directions if any. 

\section*{Acknowledgements}

I am thankful to Professor Hosseini for introducing us to the concept of Fourier Transforms and explaining
their significance. I am also grateful to the YouTube Channel "3Blue1Brown" by Grant Sanderson for providing
a visual explanation for the intuition behind Fourier Transforms/series, and how they can be used to find 
signals within data. 

I am very thankful to my peers taking the class alongside me, they have helped me understand the material as well
as provide a reference to compare my results against. I interacted with them both through Canvas discussion boards
as well as Discord chat. 
% Make sure you you clearly state any help you received including collaborations 
% with your peers. Help from TAs or other mentors, professors, etc that helped you 
% with your assignment. Here's a formal example: 

% The author is thankful to Prof. X for useful discussions about the QR algorithm. 
% We are also thankful to Dr. Strange for suggesting the JAX software package for 
% automatic differentiation. Furthermore, our peer Jean Grey was helpful in 
% implementation of spectral clustering in Python.

\bibliographystyle{abbrv}
 % make sure this matches the .bib file for your corresponding document. You also have to maintain your references in the .bib file 
\end{document}
